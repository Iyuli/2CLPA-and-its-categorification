\documentclass[11pt]{article}
% ---------------- 导言区 ----------------
% cspell:words categorification
% ---------------- 宏包 ----------------
\usepackage[T1]{fontenc}
\usepackage[utf8]{inputenc}
\usepackage[english]{babel}
\usepackage{csquotes}                  % biblatex 推薦搭配

\usepackage{fix-cm}                     % 允許 Computer Modern 可縮放尺寸

\usepackage{amsmath,mathtools,amsthm}    % 更完整的數學環境和修正
\usepackage{amssymb,latexsym}           % 已有的符號包
\usepackage{bm}                         % 粗體數學符號 (\bm)
\usepackage{mathrsfs}                   % \mathscr 字體
\AtBeginDocument{%
  \DeclareFontShape{U}{rsfs}{m}{n}{%
    <-6> rsfs5%
    <6-8> rsfs7%
    <8-> rsfs10%
  }{}%
}
\usepackage{newtxtext}                  % 更現代的文字字體(替代 times)
\usepackage{newtxmath}                  % 數學字體
\usepackage{microtype}                  % 排版改進
\usepackage{xcolor}                     % 顏色支援
\usepackage{graphicx}                   % 圖片
\usepackage{caption,subcaption}         % 子圖與標註
\usepackage{enumitem}                   % 列表控制
\usepackage{siunitx}                    % 單位與數字格式
\usepackage[all]{xy}
\usepackage{tikzit}
\input{chg2.tikzstyles}

\usepackage[backend=biber]{biblatex}

% biblatex 已在原文件中設定,下面確保不顯示 date/url 等欄位
\AtEveryBibitem{%
  \clearfield{month}%
  \clearfield{year}%
  \clearfield{date}%
  \clearfield{urldate}%
  \clearfield{url}%
}

% hyperref 要放在大多數宏包之後(最後但在自訂命令之前)
\usepackage[
  hidelinks,
  unicode=true,
  pdfauthor={Yiyu Li},
  pdftitle={2-cyclic Laurent polynomial algebras and their Categorifications}
]{hyperref}
\usepackage{bookmark} % 改善目錄書籤

% ---------------- 页面设置 ----------------
\textheight 230mm
\textwidth 150mm
\hoffset -16mm
\voffset -16mm

% ---------------- 定理环境 ----------------
\newtheorem{Thm}{Theorem}[section]
\newtheorem{Lem}[Thm]{Lemma}
\newtheorem{Cor}[Thm]{Corollary}
\newtheorem{Prop}[Thm]{Proposition}
\newtheorem{Conj}[Thm]{Conjecture}
\newtheorem{Def}[Thm]{Definition}
\newtheorem{example}[Thm]{Example}
\newtheorem{remark}[Thm]{Remark}

% ---------------- 自定义命令 ----------------
% ---------------- mathbb ----------------
\newcommand{\bA}{\mathbb{A}}
\newcommand{\bB}{\mathbb{B}}
\newcommand{\bC}{\mathbb{C}}
\newcommand{\bD}{\mathbb{D}}
\newcommand{\bE}{\mathbb{E}}
\newcommand{\bF}{\mathbb{F}}
\newcommand{\bG}{\mathbb{G}}
\newcommand{\bH}{\mathbb{H}}
\newcommand{\bI}{\mathbb{I}}
\newcommand{\bJ}{\mathbb{J}}
\newcommand{\bK}{\mathbb{K}}
\newcommand{\bL}{\mathbb{L}}
\newcommand{\bM}{\mathbb{M}}
\newcommand{\bN}{\mathbb{N}}
\newcommand{\bO}{\mathbb{O}}
\newcommand{\bP}{\mathbb{P}}
\newcommand{\bQ}{\mathbb{Q}}
\newcommand{\bR}{\mathbb{R}}
\newcommand{\bS}{\mathbb{S}}
\newcommand{\bT}{\mathbb{T}}
\newcommand{\bU}{\mathbb{U}}
\newcommand{\bV}{\mathbb{V}}
\newcommand{\bW}{\mathbb{W}}
\newcommand{\bX}{\mathbb{X}}
\newcommand{\bY}{\mathbb{Y}}
\newcommand{\bZ}{\mathbb{Z}}
% ---------------- mathcal ----------------
\newcommand{\cA}{\mathcal{A}}
\newcommand{\cB}{\mathcal{B}}
\newcommand{\cC}{\mathcal{C}}
\newcommand{\cD}{\mathcal{D}}
\newcommand{\cE}{\mathcal{E}}
\newcommand{\cF}{\mathcal{F}}
\newcommand{\cG}{\mathcal{G}}
\newcommand{\cH}{\mathcal{H}}
\newcommand{\cI}{\mathcal{I}}
\newcommand{\cJ}{\mathcal{J}}
\newcommand{\cK}{\mathcal{K}}
\newcommand{\cL}{\mathcal{L}}
\newcommand{\cM}{\mathcal{M}}
\newcommand{\cN}{\mathcal{N}}
\newcommand{\cO}{\mathcal{O}}
\newcommand{\cP}{\mathcal{P}}
\newcommand{\cQ}{\mathcal{Q}}
\newcommand{\cR}{\mathcal{R}}
\newcommand{\cS}{\mathcal{S}}
\newcommand{\cT}{\mathcal{T}}
\newcommand{\cU}{\mathcal{U}}
\newcommand{\cV}{\mathcal{V}}
\newcommand{\cW}{\mathcal{W}}
\newcommand{\cX}{\mathcal{X}}
\newcommand{\cY}{\mathcal{Y}}
\newcommand{\cZ}{\mathcal{Z}}
% ---------------- mathfrak ----------------
\newcommand{\fg}{\mathfrak{g}}
% ---------------- AlgRepSign ----------------
\newcommand{\Hom}{\operatorname{Hom}}
\newcommand{\Ext}{\operatorname{Ext}}
\newcommand{\End}{\operatorname{End}}
\newcommand{\Tor}{\operatorname{Tor}}
\newcommand{\dimv}{\underline{\dim}}
\newcommand{\Gr}{\operatorname{Gr}}

% ensure scalable fonts and provide math script alphabet to avoid "U/rsfs" font-shape warnings
\usepackage[T1]{fontenc}
\usepackage{lmodern}
\usepackage{mathrsfs}
\DeclareMathAlphabet{\mathscr}{U}{rsfs}{m}{n}
\addbibresource{2-cyclicLaurentPolynomialAlgebras.bib}
% ---------------- 文档开始 ----------------
\begin{document}

\title{2-cyclic Laurent polynomial algebras and their Categorifications}
\author{Yiyu Li}
\date{}
\maketitle

\begin{abstract}
In this paper, we explore the 2-cyclic quiver Laurent polynomial algebra and its categorification. Inspired by the framework of cluster algebras, the study begins by introducing the concept of 2-cyclic quivers and their mutation. We define 2-cyclic Laurent polynomial algebra is established and prove that the Laurent monomial phenomenon behaved on 2-cyclic Laurent polynomial algebra. We further classify 2-cyclic Laurent polynomial algebras in the finite type case and extends essential concepts from cluster algebras, such as $c$-vectors, $g$-vectors, and $F$-polynomials, to the 2-cyclic Laurent polynomial algebra framework, as a consequence, We prove the sign-coherence of $c$-vectors of 2-cyclic Laurent polynomial algebra.. Moreover, we establish the connections between 2-cyclic Laurent polynomial algebras and cluster algebras.

We also expand the Derksen-Weyman-Zelevinsky theory of quivers with potential to the 2-cyclic quivers. A mutation theory for 2-cyclic quivers with potentials is developed, and we prove that the mutation is an involution and that finite-dimensionality is a mutation invariant. Additionally, we show the existence of a well-behaved Jacobian-finite potential for the $\bA_n$-type 2-cyclic quiver, and furthermore, proves that the potential is also non-degenerate when $n \leq 3$.

In the final part of this thesis, combining the theory of  generalized cluster categories, the thesis investigates the generalized cluster category of a 2-cyclic quiver with potential of $\bA_n$-type. By constructing the corresponding covering functor over the $\mathbb{Z}_3$-cover of the 2-cyclic quiver with potential, the reachable cluster tilting objects are identified. In addition, we introduce the definition of  2-cyclic Caldero-Chapoton (CC) formula:
\begin{equation}
CC(X)=X^{-\fg_X}\sum_{e\in\bN^n}\chi\bigl(\Gr_e(H^0(X))\bigr)y^e,
\end{equation}
and it is proven that this map categorifies the $g$-vectors of the $\bA_n$-type 2-cyclic Laurent polynomial algebra, ensuring that the $\bA_n$-type generalized cluster category is cluster-tilting finite. For the categorification of the $F$-polynomials of 2-cyclic Laurent polynomial algebras, we prove that the F-polynomials of indecomposable $\tau$-rigid modules for $\bA_n$-type Jacobian algebras correspond to the $F$-polynomials of the associated 2-cyclic quiver when $n \leq 3$.

\end{abstract}

\section{Introduction}
 Foundational aspects of cluster algebras can be found in \cite{fominClusterAlgebrasII2003}.

\section{2-cyclic Laurent Polynomial Algebras}
\subsection{2-cyclic quivers and their mutation}
\subsection{2-cyclic Laurent polynomial algebras}
\subsection{\texorpdfstring{$c$-vectors, $g$-vectors, and $F$-polynomials}{c-vectors, g-vectors, and F-polynomials}}
\subsection{Classification of 2-cyclic Laurent polynomial algebras}

\section{2-cyclic quivers with potentials}
\subsection{Mutation of 2-cyclic quivers with potentials}

\printbibliography

\end{document}
