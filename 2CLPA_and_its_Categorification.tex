\documentclass[11pt]{article}

% ---------------- Package Imports ----------------
\usepackage{amsmath}
\usepackage{amssymb}
\usepackage{amsthm}
\usepackage[english]{babel}
\usepackage[backend=biber,backref=true]{biblatex}
\usepackage{cases}
\usepackage{colonequals}
\usepackage{csquotes}
\usepackage{enumerate}
\usepackage{etoolbox}
\usepackage{fancyhdr}
\usepackage{float}
\usepackage[T1]{fontenc}
\usepackage{geometry}
\usepackage{graphicx}
\usepackage[colorlinks=true,linkcolor=Maroon,citecolor=LimeGreen,linktocpage=true]{hyperref}
%\usepackage[utf8]{inputenc}
\usepackage{latexsym}
\usepackage{mathpazo}
\usepackage{mathrsfs}
\usepackage{mathtools}
\usepackage{microtype}
\UseMicrotypeSet[protrusion]{basicmath} % enable protrusion for math
\usepackage{multicol}
\usepackage{setspace}
\usepackage{soul}
\usepackage{stmaryrd}
\usepackage{tikz}
\usepackage{tikz-cd}
\usepackage[dvipsnames]{xcolor}
\usepackage[all]{xy}


% ---------------- Page Layout ----------------
\geometry{left=2.5cm,right=2cm,top=2.5cm,bottom=3.3cm}

% ---------------- Bibliography ----------------
\addbibresource{2-cyclicLaurentPolynomialAlgebras.bib}

% ---------------- Numbering Conventions ----------------
\numberwithin{equation}{section}

% ---------------- Macros ----------------
% Blackboard bold letters
\newcommand{\bA}{\mathbb{A}}
\newcommand{\bB}{\mathbb{B}}
\newcommand{\bC}{\mathbb{C}}
\newcommand{\bD}{\mathbb{D}}
\newcommand{\bE}{\mathbb{E}}
\newcommand{\bF}{\mathbb{F}}
\newcommand{\bG}{\mathbb{G}}
\newcommand{\bK}{\mathbb{K}}
\newcommand{\bL}{\mathbb{L}}
\newcommand{\bN}{\mathbb{N}}
\newcommand{\bP}{\mathbb{P}}
\newcommand{\bQ}{\mathbb{Q}}
\newcommand{\bX}{\mathbb{X}}
\newcommand{\bZ}{\mathbb{Z}}

% Calligraphic letters
\newcommand{\cA}{\mathcal{A}}
\newcommand{\cB}{\mathcal{B}}
\newcommand{\cC}{\mathcal{C}}
\newcommand{\cD}{\mathcal{D}}
\newcommand{\cE}{\mathcal{E}}
\newcommand{\cF}{\mathcal{F}}
\newcommand{\cG}{\mathcal{G}}
\newcommand{\cH}{\mathcal{H}}
\newcommand{\cI}{\mathcal{I}}
\newcommand{\cJ}{\mathcal{J}}
\newcommand{\cK}{\mathcal{K}}
\newcommand{\cL}{\mathcal{L}}
\newcommand{\cM}{\mathcal{M}}
\newcommand{\cN}{\mathcal{N}}
\newcommand{\cO}{\mathcal{O}}
\newcommand{\cP}{\mathcal{P}}
\newcommand{\cQ}{\mathcal{Q}}
\newcommand{\cR}{\mathcal{R}}
\newcommand{\cS}{\mathcal{S}}
\newcommand{\cT}{\mathcal{T}}
\newcommand{\cU}{\mathcal{U}}
\newcommand{\cV}{\mathcal{V}}
\newcommand{\cW}{\mathcal{W}}
\newcommand{\cX}{\mathcal{X}}
\newcommand{\cY}{\mathcal{Y}}
\newcommand{\cZ}{\mathcal{Z}}


% Greek shortcuts
\newcommand{\La}{\Lambda}
\newcommand{\vf}{\varphi}

% Operators and functors
\newcommand{\add}{\operatorname{add}}
\newcommand{\Aut}{\operatorname{Aut}}
\newcommand{\coker}{\operatorname{Coker}}
\newcommand{\End}{\operatorname{End}}
\newcommand{\Ext}{\operatorname{Ext}}
\newcommand{\gl}{\operatorname{gl.dim}}
\newcommand{\Gr}{\operatorname{Gr}}
\newcommand{\Hom}{\operatorname{Hom}}
\newcommand{\im}{\operatorname{Im}}
\newcommand{\injd}{\operatorname{inj.dim}}
\newcommand{\Iso}{\operatorname{Iso}}
\newcommand{\modu}{\operatorname{mod}}
\newcommand{\pd}{\operatorname{proj.dim}}
\newcommand{\rad}{\operatorname{rad}}
\newcommand{\soc}{\operatorname{soc}}
\newcommand{\supp}{\operatorname{supp}}

% Miscellaneous symbols
\newcommand{\D}{\partial}
\newcommand{\fg}{\mathfrak{g}}
\newcommand{\HG}{\mathrm{H}}
\newcommand{\la}{\langle}
\newcommand{\Proj}{\mathbb{P}}
\newcommand{\qbinom}[2]{\begin{bmatrix} #1\\#2 \end{bmatrix}}
\newcommand{\ra}{\rangle}

% Allow multiline displays to break across pages
\allowdisplaybreaks

% ---------------- Theorem Environments ----------------
\newtheorem{theorem}{Theorem}[section]
\newtheorem*{acknowledgement}{Acknowledgements}
\newtheorem{algorithm}[theorem]{Algorithm}
\newtheorem{axiom}[theorem]{Axiom}
\newtheorem{case}[theorem]{Case}
\newtheorem{claim}[theorem]{Claim}
\newtheorem{conclusion}[theorem]{Conclusion}
\newtheorem{condition}[theorem]{Condition}
\newtheorem{conjecture}[theorem]{Conjecture}
\newtheorem{construction}[theorem]{Construction}
\newtheorem{corollary}[theorem]{Corollary}
\newtheorem{criterion}[theorem]{Criterion}
\newtheorem{definition}[theorem]{Definition}
\newtheorem{example}[theorem]{Example}
\newtheorem{exercise}[theorem]{Exercise}
\newtheorem{lemma}[theorem]{Lemma}
\newtheorem{notation}[theorem]{Notation}
\newtheorem{problem}[theorem]{Problem}
\newtheorem{proposition}[theorem]{Proposition}
\newtheorem{solution}[theorem]{Solution}
\newtheorem{summary}[theorem]{Summary}
\newtheorem*{thm}{Theorem}

\theoremstyle{remark}
\newtheorem{remark}[theorem]{Remark}

\begin{document}
\onehalfspacing
\input xy
\xyoption{all}

% ---------------- Document Metadata ----------------
\title{\textbf{2-cyclic Laurent Polynomial Algebras \\and their categorifications}}
\author{Yiyu Li}
\date{}

% ---------------- Author Details ----------------
\newcommand{\AuthorAffiliation}{Yiyu Li, School of Mathematical Science, Beijing Normal University, Chengdu 610064, P.R.China}
\newcommand{\AuthorEmail}{\texttt{liyiyumath@gmail.com}}

\AtEndDocument{%
  \par\bigskip
  \begin{center}
    \small \AuthorAffiliation\par
    \AuthorEmail
  \end{center}%
}

\maketitle

\begin{abstract}

In this paper, we explore the 2-cyclic quiver Laurent polynomial algebra and its categorification. Inspired by the framework of cluster algebras, the study begins by introducing the concept of 2-cyclic quivers and their mutation. We define 2-cyclic Laurent polynomial algebra is established and prove that the Laurent monomial phenomenon behaved on 2-cyclic Laurent polynomial algebra. We further classify 2-cyclic Laurent polynomial algebras in the finite type case and extends essential concepts from cluster algebras, such as $c$-vectors, $g$-vectors, and $F$-polynomials, to the 2-cyclic Laurent polynomial algebra framework, as a consequence, We prove the sign-coherence of $c$-vectors of 2-cyclic Laurent polynomial algebra. Moreover, we establish the connections between 2-cyclic Laurent polynomial algebras and cluster algebras.

We also expand the Derksen-Weyman-Zelevinsky theory of quivers with potential to the 2-cyclic quivers. A mutation theory for 2-cyclic quivers with potentials is developed, and we prove that the mutation is an involution and that finite-dimensionality is a mutation invariant. Additionally, we show the existence of a well-behaved Jacobian-finite potential for the $\bA_n$-type 2-cyclic quiver, and furthermore, proves that the potential is also non-degenerate when $n \leq 3$.

In the final part of this thesis, combining the theory of  generalized cluster categories, the thesis investigates the generalized cluster category of a 2-cyclic quiver with potential of $\bA_n$-type. By constructing the corresponding covering functor over the $\mathbb{Z}_3$-cover of the 2-cyclic quiver with potential, the reachable cluster tilting objects are identified. In addition, we introduce the definition of  2-cyclic Caldero-Chapoton (CC) formula:
\begin{equation}
CC(X)=X^{-\fg_X}\sum_{e\in\bN^n}\chi\bigl(\Gr_e(H^0(X))\bigr)y^e,
\end{equation}
and it is proven that this map categorifies the $g$-vectors of the $\bA_n$-type 2-cyclic Laurent polynomial algebra, ensuring that the $\bA_n$-type generalized cluster category is cluster-tilting finite. For the categorification of the $F$-polynomials of 2-cyclic Laurent polynomial algebras, we prove that the F-polynomials of indecomposable $\tau$-rigid modules for $\bA_n$-type Jacobian algebras correspond to the $\cF$-polynomials of the associated 2-cyclic quiver when $n \leq 3$.

\end{abstract}

\section{Introduction}\label{sec: intro}

\cite{fominClusterAlgebrasIV2007}

\section{Preliminaries}\label{sec: prelim}
   
   \subsection{Quivers and Quiver Representations}\label{subsec: quiverRep}
   
   \subsection{Cluster Algebras}\label{subsec: clusterAlg}
   
   \subsection{Ginzburg dg Algebras and Generalized Cluster Categories}\label{subsec: genClusterCat}
   
   \subsection{Covering Theory}\label{subsec: coveringTheory}

\section{2-cyclic Laurent Polynomial Algebras}\label{sec: 2CLPA}
    
    \subsection{2-cyclic Quivers and their Mutations}\label{subsec: 2cyclicQuiverMut}

    \subsection{2-cyclic Laurent Polynomial Algebras}\label{subsec: 2CLPA}
    
    \subsection{$c$-vectors, $g$-vectors and $F$-polynomials}\label{subsec: cgfVectors}

    \subsection{Finite Type Classification}\label{subsec: finiteTypeClass}
    
    \subsection{Relation with Cluster Algebras}\label{subsec: relationClusterAlg}

\section{2-cyclic Quivers with Potentials and Their Cluster Categories}\label{sec: 2cyclicQPs}
    
    \subsection{2-cyclic Quivers with Potential and their Mutations}\label{subsec: 2cyclicQPMut}
    
    \subsection{Finite-dimensionality and Non-degeneracy}\label{subsec: finiteDimNondeg}
    
    \subsection{Generalized Cluster Categories of 2-cyclic QPs}\label{subsec: genClusterCat2cyclicQPs}

    \subsection{Cluster Tilting Objects and Categorification of $g$-vectors}\label{subsec: clusterTiltingCatgVectors}

    \subsection{Categorification of $F$-polynomials}\label{subsec: catFpolynomials}
\begin{acknowledgement}

\end{acknowledgement}


\printbibliography


\end{document}
